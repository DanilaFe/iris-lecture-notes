\section{Case Study: Types and Abstraction: Logical Relations in Iris}
\label{sec:logical-relations}

This section is still work-in-progress. It is closely based on~\cite{timany:logrel-in-iris};
indeed large parts of this section are taken directly from~\cite{timany:logrel-in-iris},
with permission from the authors.

\newcommand{\EqTyp}{\mathsf{EqType}}
\newcommand{\TheLang}{\(\mathsf{F}_{\mu, \mathit{ref}, \mathit{conc}}\)}
\newcommand{\SAFE}{\mathit{Safe}}
\newcommand{\lectx}{\le_{\mathit{ctx}}}
\newcommand{\lelog}{\le_{\mathit{log}}}
\newcommand{\valC}{u}
\newcommand{\len}{\mathrm{length}}

%%%%%%%%%%%%%%%%%%%%%%%%%%%%%%%%%%%%%%%%%%%%%%%%%%%%%%%%%%%%%%%%
% Logical relations
%%%%%%%%%%%%%%%%%%%%%%%%%%%%%%%%%%%%%%%%%%%%%%%%%%%%%%%%%%%%%%%%

\newcommand{\semVtype}[3]{ \llbracket #1 \vdash #2 \rrbracket_{#3} }
\newcommand{\semEtype}[3]{ \llbracket #1 \vdash #2 \rrbracket^{\mathcal{E}}_{#3} }
\newcommand{\semGtype}[3]{ \llbracket #1 \vdash #2 \rrbracket^{\mathcal{G}}_{#3} }

\newcommand{\cfgg}{\rho}
\newcommand{\SpecConf}{\textrm{SpecConf}}
\newcommand{\SpecCtx}{\textrm{SpecCtx}}

\renewcommand{\qedsymbol}{$\square$}

So far we have used Iris for specifying and reasoning about programs written in the
\emph{untyped} programming language \proglang. 
In this section we will consider a \emph{typed} programming language,
\TheLang,  and show how we can use Iris to give semantic interpretations
of the types of \TheLang.  

In the words of Reynolds~\cite{reynolds:types}, a type system ``is a syntactic
discipline for enforcing levels of abstraction''.
One of the fundamental properties of a type system is type
soundness, \ie\ that ``syntactically well-typed programs don't go wrong''~\cite{Milner:78}.
In contrast to the properties we have considered earlier in these
notes, where we so far have focused on properties of individual
programs, type soundness is a language property: it is a property that 
depends on the operational semantics and the type system for the whole
programming language and it is a property that one proves once and 
for all for a programming language. 
There are different approaches to proving type soundness, most notably
the syntactic approach based on progress and preservation lemmas (for
textbook treatments, see, e.g.,~\cite{pierce:tapl,harper:practical-foundations}), and 
the semantic approach where one gives a semantic model of the types 
and proves that any syntactically well-typed program is in the
semantic interpretation of its type. 

An advantage of the semantic approach is that it gives a semantic account of the invariants
enforced by the syntactic type system. This means that it allows one
to \emph{combine} syntactically well-typed programs with programs which are
semantically, but not necessarily syntactially, well-typed. This is
important in practise since most realistic statically typed programming
languages include some facility for interacting with programs that are
not syntactically well typed (e.g., through foreign function call
interfaces, or through an ``unsafe'' construct~\cite{}). 

An advantage of the syntactic approach using progress and preservation
lemmas is that it scales well to advanced type systems including
impredicative polymorphism, recursive types, and general reference
types. In contrast, an often-cited challenge with the semantic
approach is that it is non-trivial to define semantic interpretations of advanced type
systems (because the semantic models need to support recursive
definitions and impredicative invariants). 

In this section we show that \emph{by interpreting types as 
Iris propositions it is straightforward to give a semantic
interpretation of types}, and prove that all syntactically well-typed
programs are in the appropriate semantic interpretation. Moreover,
we can also use Iris to show semantic well-typedness of 
programs that are not syntactically well-typed.

The key point is that we can exploit Iris features such as invariants, persistence, and the possibility of defining predicates by guarded recursion to give a simple inductive definition of the interpretation of all the types of \TheLang, which include challenging types such as impredicative polymorpic types, recursive types, and general reference types.
We focus on type soundness in Subsection~\ref{sec:unary-logical-relation}.

Another advantage of the semantic approach is that it can be
generalized to give a \emph{relational interpretation} of types, which
is useful for proving contextual refinement and data abstraction results.
It is well-known that relational models for showing contextual refinement are also
non-trivial to construct. Moreover, it is particularly challenging to
construct relational models which are sufficiently powerful to allow
one to prove relatedness of programs that use state and concurrency in
very different ways. Using Iris, however, we can addresss both of
these challenges. Indeed, in~\cite{timany:logrel-in-iris} a relational
interpretation of the types of \TheLang\ is also defined, by
interpreting types as Iris relations. Moreover, it is shown that the
relational interpretation is expressive enough to allow one to prove
challenging examples of program refinements. For example, it is proved
that a fine-grained concurrent stack module is a contextual refinement of a
coarse-grained stack module. 
In light of the fact that we so far only used Iris for reasoning about a
single program at a time, it is perhaps somewhat surprising that we can also
use Iris to relate two different programs.  Thus, in
Subsection~\ref{sec:binary-logical-relation} we sketch the key idea of
how this can be done. 
However, we do not include a full description of the relational
model, but instead refer the reader to~\cite{timany:logrel-in-iris}. 


\subsection{The language \texorpdfstring{\TheLang}{Fmurefconc}}
\TheLang\, is mostly as \proglang, the main exception being that it is a typed languages. We have all the same values and expressions as before with a few additions: 
\begin{itemize}
\item \langkw{fold} and \langkw{unfold} respectively fold and unfold expressions of recursive types.
\item $\Lambda$ is a type-level lambda abstraction.
\end{itemize}
\begin{figure}[htbp]
\textbf{Syntax}
\begin{displaymath}
  \begin{array}{lrcl}
\Val\quad
& \val & \bnfdef{}&
 \cdots
 \ALT \langkw{fold}\, \val
 \ALT \Lambda \, \expr
\\
%
\Expr \quad
& \expr & \bnfdef{} &
 \cdots
 \ALT \langkw{fold} \, \expr
 \ALT \langkw{unfold} \, \expr
 \ALT \Lambda \, \expr 
 \ALT \expr \, \_
\\
%
\ECtx\quad
& E & \bnfdef{}& 
 \cdots
 \ALT \langkw{fold} \, E
 \ALT \langkw{unfold} \, E
 \ALT E \, \_ 
 \\
%
 Types \quad & \typ & \bnfdef{}& 
 X
 \ALT 1
 \ALT \nat
 \ALT \mathbb{B}
 \ALT \typ \to \typ
 \ALT \All X. \typ
 \ALT \typ \times \typ
 \ALT \typ + \typ
 \ALT \mu X. \typ
 \ALT ref \typ
\\
%
\end{array}
\end{displaymath}
\end{figure}
We have the same reductions as \proglang{} with the addition of the following two pure reductions:

\begin{align*}
 \left( \Lambda \, \expr \right)\, \_ &\stepstopure \expr \\
 \langkw{unfold}\, (\langkw{fold}\, \val) &\stepstopure \val
\end{align*}


As the weakest precondition is defined from the operational semantics, we get the same wp-rules as in Figure~\ref{fig:wp-rules} with the addition of the following two rules, corresponding to our two new reductions:

\begin{mathpar}
\inferH{wp-Tlam}{}{\later \wpre{\expr}[\mask]{\pred}\proves \wpre{(\TLam \expr)~\_}[\mask]{\pred}}
\and
\inferH{wp-fold}{}{\later \wpre{\val}[\mask]{\pred}\proves \wpre{\unfold(\fold \val)}[\mask]{\pred}}
\end{mathpar}

To make \TheLang\, a typed language, we naturally need some typing rules. These are fairly standard (see~\cite{pierce:tapl,harper:practical-foundations})
and given in Figure~\ref {fig:logrel-iris-typing-rules}.


\begin{figure}
\begin{mathparpagebreakable}
  \inferH{T-var}{\var : \typ \in \env}{\typed{\Tenv}{\env}{\var}{\typ}}
  \and
  \inferH{T-Unit}{}{\typed{\Tenv}{\env}{\TT}{\Tunit}}
  \and
  \inferH{T-Nat}{}{\typed{\Tenv}{\env}{n}{\Tnat}}
  \and
  \inferH{T-Bool}{\val \in \set{\True, \False}}{\typed{\Tenv}{\env}{\val}{\Tbool}}
  \and
  \inferH{T-rec}{\typed{\Tenv}{\env, \var : \typ, f : \typ \to \typ'}{\expr}{\typ'}}
  {\typed{\Tenv}{\env}{\Rec f \var = \expr}{\typ \to \typ'}}
  \and
  \inferH{T-app}{\typed{\Tenv}{\env}{\expr_1}{\typ \to \typ'} \and \typed{\Tenv}{\env}{\expr_2}{\typ}}
  {\typed{\Tenv}{\env}{\expr_1~\expr_2}{\typ'}}
  \and
  \inferH{T-tlam}{\typed{\Tenv, \tvar}{\env}{\expr}{\typ}}
  {\typed{\Tenv}{\env}{\TLam \expr}{\Tall \tvar. \typ}}
  \and
  \inferH{T-tapp}{\typed{\Tenv}{\env}{\expr}{\Tall \tvar. \typ}}
  {\typed{\Tenv}{\env}{\expr~\_}{\subst{\typ}{\tvar}{\typ'}}}
  \and
  \inferH{T-if}{\typed{\Tenv}{\env}{\expr}{\Tbool} \and
    \typed{\Tenv}{\env}{\expr_{i}}{\typ} \and i \in \set{1, 2}}
  {\typed{\Tenv}{\env}{\If \expr then \expr_1 \Else \expr_2}{\typ}}
  \and
  \inferH{T-pair}{\typed{\Tenv}{\env}{\expr_1}{\typ_1} \and \typed{\Tenv}{\env}{\expr_2}{\typ_2}}
  {\typed{\Tenv}{\env}{(\expr_1, \expr_2)}{\typ_1 \times \typ_2}}
  \and
  \inferH{T-proj}{\typed{\Tenv}{\env}{\expr}{\typ_1 \times \typ_2} \and i \in \set{1, 2}}
  {\typed{\Tenv}{\env}{\Proj{i} \expr}{\typ_{i}}}
  \and
  \inferH{T-inj}{\typed{\Tenv}{\env}{\expr}{\typ_{i}} \and i \in \set{1, 2}}
  {\typed{\Tenv}{\env}{\Inj{i} \expr}{\typ_1 + \typ_2}}
  \and
  \inferH{T-match}{\typed{\Tenv}{\env}{\expr}{\typ_1 + \typ_2} \and
    \typed{\Tenv}{\env, \var : \typ_{i}}{\expr_{i}}{\typ_3} \and i \in \set{1, 2}}
  {\typed{\Tenv}{\env}{\MatchS \expr with \Inj{i} \var => \expr_i end}{\typ_3}}
  \and
  \inferH{T-fold}{\typed{\Tenv}{\env}{\expr}{\subst{\typ}{\tvar}{\Tmu \tvar. \typ}}}
  {\typed{\Tenv}{\env}{\fold \expr}{\Tmu \tvar. \typ}}
  \and
  \inferH{T-unfold}{\typed{\Tenv}{\env}{\expr}{\Tmu \tvar. \typ}}
  {\typed{\Tenv}{\env}{\unfold \expr}{\subst{\typ}{\tvar}{\Tmu \tvar. \typ}}}
  \and
  \inferH{T-alloc}{\typed{\Tenv}{\env}{\expr}{\typ}}
  {\typed{\Tenv}{\env}{\Ref(\expr)}{\Tref(\typ)}}
  \and
  \inferH{T-load}{\typed{\Tenv}{\env}{\expr}{\Tref(\typ)}}
  {\typed{\Tenv}{\env}{\deref \expr}{\typ}}
  \and
  \inferH{T-store}{\typed{\Tenv}{\env}{\expr_1}{\Ref(\typ)} \and \typed{\Tenv}{\env}{\expr_2}{\typ}}
  {\typed{\Tenv}{\env}{\expr_1 \gets \expr_2}{\Tunit}}
  \and
  \inferH{T-CAS}{\typed{\Tenv}{\env}{\expr_1}{\Ref(\typ)} \and \typed{\Tenv}{\env}{\expr_2}{\typ}
    \and \typed{\Tenv}{\env}{\expr_3}{\typ} \and \EqTyp(\typ)}
  {\typed{\Tenv}{\env}{\CAS(\expr_1, \expr_2, \expr_3)}{\Tbool}}
  \and
  \inferH{EqTyp-unit}{}{\EqTyp(\Tunit)}
  \and
  \inferH{EqTyp-nat}{}{\EqTyp(\Tnat)}
  \and
  \inferH{EqTyp-bool}{}{\EqTyp(\Tbool)}
  \and
  \inferH{EqTyp-ref}{}{\EqTyp(\Tref(\typ))}
  \and
  \inferH{T-fork}{\typed{\Tenv}{\env}{\expr}{\typ}}
  {\typed{\Tenv}{\env}{\Fork{\expr}}{\Tunit}}
\end{mathparpagebreakable}
\caption{The typing rules of \TheLang{}.}
\label{fig:logrel-iris-typing-rules}
\end{figure}

Our goal is to prove type safety, hence we need a notion of safety. Here we will use the following definition:

\paragraph{Safety} We say a program $\expr$ is safe, written
$\SAFE(\expr)$, if it does not get stuck.
In more detail, $\expr$ is safe if, for all expressions $\expr'$ that $\expr$ or one
of the child thread of $\expr$ has evaluated to, $\expr'$ is either a value or it
can be evaluated further by making a head step, or by forking a thread.
This is formally defined as follows:
\begin{align*}
  \SAFE(\expr) \eqdef{}
  & \All \expr', \vv{\expr_1}, \vv{\expr_2}, \state.\; (\emptyset,
    \expr) \tpstep ^{*} (\state, \vv{\expr_1};\expr';\vv{\expr_2}) \Rightarrow \\
  & \expr' \in \Val \lor (\exists \expr'', \state'.\; (\state, \vv{\expr_1};\expr';\vv{\expr_2})
    \tpstep (\state', \vv{\expr_1};\expr'';\vv{\expr_2})) \lor \\
  & (\exists \expr'', \state', \expr_3.\; (\state, \vv{\expr_1};\expr'; \vv{\expr_2})
    \tpstep (\state', \vv{\expr_1};\expr'';\vv{\expr_2}; \expr_3))
\end{align*}


\subsection{Unary Logical relation}
\label{sec:unary-logical-relation}

% Note: This entire subsection is taken almost verbatim
% from~\cite{timany:logrel-in-iris}.

In this section we define a unary logical relations model for
\TheLang\, and use it to prove the type soundness theorem.  That is,
for each type we define a logical relation. We show that each
well-typed program is in the relation for its type. Furthermore, we
show that programs in the logical relation for a type have
well-defined behavior, \ie\ they do not get stuck. The type soundness
theorem is a direct consequence of these two facts.

We define the logical relations in three stages. We first define a
relation for \emph{closed} values of a type by induction on types. We
then use the value relations to define relations on \emph{closed}
expressions. Intuitively, a closed expression is in the relation for a
type $\typ$ if it is a computation that results in a value that is in
the value relation for the type $\typ$. Finally, we define \emph{the
  logical relation} for (open) expressions based on the expression
relations and the value relations above.

\TheLang\, features polymorphism. Hence, types can have free type
variables. Thus, we index the relations on closed values and
expressions with a map, $\semenv$, which assigns a semantic type (a
value relation) to each free type variable. That is, for each type
$\typ$ we define the value relation
$\semVtype{\Tenv}{\typ}{\semenv} : \Val \to \Prop$ where
$\semenv : \Tenv \to \Val \to \Prop$. The full definition of the
value relations for types is given in
Figure~\ref{fig:logrel-in-iris-unary-valrel}. We will discuss them in
detail below. Before that we discuss how value relations are extended
to expression relation on closed and subsequently open expressions.

Intuitively, a closed expression is in the expression relation for a
type $\typ$ if it computes a result that is in the value relation for
the type $\typ$. We define the expression relation,
$\semEtype{\Tenv}{\typ}{\semenv} : \Expr \to \Prop$, using 
weakest preconditions, as follows:
\[
\semEtype{\Tenv}{\typ}{\semenv}(\expr) \eqdef{} \wpre{\expr}{\semVtype{\Tenv}{\typ}{\semenv}}
\]

In order to formally define the logical relation for open expressions
we first define a relation, $\semGtype{\Tenv}{\cdot}{\semenv}$, for
typing contexts. Intuitively, a list a values, $\vv{\val}$, is in the
relation for a typing context, $\env$, if each value in $\vv{\val}$ is
in the value relation for the type corresponding to it in $\env$. The
formal definition of the typing-context relation is given below. Here,
$\epsilon$ is the empty list of values.
\begin{align*}
  \semGtype{\Tenv}{\cdot}{\semenv}(\epsilon) \eqdef{}
  & \top\\
  \semGtype{\Tenv}{\env, \var : \typ}{\semenv}(\vv{\val}, \valB) \eqdef{}
  & \semGtype{\Tenv}{\env}{\semenv}(\vv{\val}) \ast
    \semVtype{\Tenv}{\typ}{\semenv}(\valB)
\end{align*}

Let $\Tenv$, $\env$, $\expr$ and $\typ$ be such that all free
variables of $\expr$ are in the domain of $\env$ and all free type
variables that appear in $\env$ or $\typ$ are in $\Tenv$. Then, we
write $\semtyped{\Tenv}{\env}{\expr}{\typ}$ to express that the
expression $\expr$ is in the logical relation for type $\typ$ under
the typing contexts $\env$ and $\Tenv$. This relation is defined as
follows:
\[
  \semtyped{\Tenv}{\env}{\expr}{\typ} \eqdef{} \forall \semenv,
  \vv{\val}.\; \semGtype{\Tenv}{\env}{\semenv}(\vv{\val}) \proves
  \semEtype{\Tenv}{\typ}{\semenv}\left(\subst{\expr}{\vv{\var}}{\vv{\val}}\right)
\]

The value relation for types are given in
Figure~\ref{fig:logrel-in-iris-unary-valrel}. One important aspect of
the type system of \TheLang\, is that it is intuitionistic, \ie\
values, e.g., function arguments, can be used multiple times. Thus, it
is crucial that the value relation of all types are \emph{persistent}
and hence duplicable. The persistence modality and the side-condition
$\persistent{\predB}$ in Figure~\ref{fig:logrel-in-iris-unary-valrel}
are added to ensure the persistence of value relations.

\begin{figure}
\begin{align*}
  \semVtype{\Tenv}{\tvar}{\semenv} \eqdef{}
  & \semenv(\tvar)\\
  \semVtype{\Tenv}{\Tunit}{\semenv}(\val) \eqdef{}
  & \val = \TT\\
  \semVtype{\Tenv}{\Tnat}{\semenv}(\val) \eqdef{}
  & \exists n \in \nat.\; \val = n\\
  \semVtype{\Tenv}{\Tbool}{\semenv}(\val) \eqdef{}
  & \val \in \set{\True, \False}\\
  \semVtype{\Tenv}{\typ_1 \times \typ_2}{\semenv}(\val) \eqdef{}
  & \exists \val_1, \val_2.\; \val = (\val_1, \val_2) \ast
    \semVtype{\Tenv}{\typ_1}{\semenv}(\val_1) \ast \semVtype{\Tenv}{\typ_2}{\semenv}(\val_2)\\
  \semVtype{\Tenv}{\typ_1 + \typ_2}{\semenv}(\val) \eqdef{}
  & \bigvee_{i \in \set{1,2}} \exists \valB.\; \val = \Inj{i} \valB \ast \semVtype{\Tenv}{\typ_i}{\semenv}(\valB)\\
  \semVtype{\Tenv}{\typ \to \typ'}{\semenv}(\val) \eqdef{}
  & \persistently \left(\forall \valB.\; \semVtype{\Tenv}{\typ}{\semenv}(\valB) \wand 
    \semEtype{\Tenv}{\typ'}{\semenv}(\val~\valB) \right) \\
  \semVtype{\Tenv}{\Tmu \tvar. \typ}{\semenv}(\val) \eqdef{}
  & \mu \predB.\; \exists \valB.\; \val = \fold \valB \land
    \later \semVtype{\Tenv}{\typ}{\semenv, \tvar \mapsto \predB}(\valB) \\
  \semVtype{\Tenv}{\Tall \tvar. \typ}{\semenv}(\val) \eqdef{}
  & \persistently\left(\forall \predB.\; \persistent{\predB} \Rightarrow
    \semEtype{\Tenv, \tvar}{\typ}{\semenv, \tvar \mapsto \predB}(\val~\_)\right)\\
  \semVtype{\Tenv}{\Tref(\typ)}{\semenv}(\val) \eqdef{}
  & \exists \loc.\; \val = \loc \land
    \knowInv{\namesp.\loc}{\exists \valB.\; \loc \mapsto \valB \ast
    \semVtype{\Tenv}{\typ}{\semenv}(\valB)}
\end{align*}
\caption{The unary value relation for types of \TheLang{}.}
\label{fig:logrel-in-iris-unary-valrel}
\end{figure}

The value relation for type variables is given by $\semenv$. A value
is in the relation for the unit type if it is $\TT$. A values is in
the relation for the type of natural numbers if it is simply a natural
number; similarly for booleans. A value is in the relation for the
product type if it is a pair of values each in their respective
types. A value of the sum type $\typ + \typ'$, on the other hand, is
either a value in the relation for $\typ$ or one in the relation for
$\typ'$. The value relation for recursive types is defined using
Iris's guarded recursive predicates. A value is in the relation for
a recursive type if it is of the form $\fold \valB$ such that the
value $\valB$ is, \emph{one step of the computation later}, in the
relation for the recursive type. Notice, however, that unfolding a
folded value takes a step of computation. A memory location is in the
relation for a reference type, $\Tref(\typ)$, if it \emph{invariantly}
stores a value that is in the value relation for $\typ$.

A value $\val$ in the relation for the function type $\typ \to \typ'$
if whenever $\val$ is applied to a value $\valB$, in the relation for
$\typ$, the resulting expression, $\val~\valB$, is in the
\emph{expression} relation for $\typ'$. A value is in the relation for
the type $\Tall \tvar. \typ$ if, when instantiated, the resulting
\emph{expression} is in the expression relation for $\typ$ where the
interpretation for $\tvar$ is taken to be any \emph{persistent}
predicate.

\begin{lemma} \label{lem:unary-interp-weaken} Let $\typ$ be a type
  such that all its free type variables appear in $\Tenv$.
  Furthermore, let $\tvar$ be a type variable such that
  $\tvar \notin \Tenv$ and let $\semenv$ be an interpretation for
  type variables in $\Tenv$. It follows that
  \[\semVtype{\Tenv}{\typ}{\semenv}(\val) \provesIff \semVtype{\Tenv,
      \tvar}{\typ}{\semenv, \tvar \mapsto \predB}(\val)
  \] for any predicate $\predB$ and value $\val$.
\end{lemma}
\begin{proof}
  By induction on the structure of $\typ$.
\end{proof}

\begin{lemma} \label{lem:unary-interp-env-weaken} Let $\env$ be a typing context
  such that all free type variables of $\env$ appear in
  $\Tenv$. Furthermore, let $\tvar$ be a type variable such that
  $\tvar \notin \Tenv$ and let $\semenv$ be an interpretation for
  type variables in $\Tenv$. It follows that
  \[\semGtype{\Tenv}{\env}{\semenv}(\vv{\val}) \provesIff \semGtype{\Tenv,
      \tvar}{\env}{\semenv, \tvar \mapsto \predB}(\vv{\val})
  \] for any predicate $\predB$ and sequence of values $\vv{\val}$.
\end{lemma}
\begin{proof}
  By induction on the length of $\env$ using Lemma~\ref{lem:unary-interp-weaken}.
\end{proof}

\begin{theorem}[Fundamental theorem of unary logical relations]\label{thm:unaryfundamental}
  All well-typed terms are in the logical relation.
  \[\text{If }~ \typed{\Tenv}{\env}{\expr}{\typ} ~\text{ then }~ \semtyped{\Tenv}{\env}{\expr}{\typ}\]
\end{theorem}
\begin{proof}
  By induction on the typing derivation. All cases follow from the
  inference rules of weakest preconditions presented in
  Section~\ref{sec:weakest-pre}. Here, we present a few cases of
  this proof.
  \begin{itemize}
  \item[--] Case \ruleref{T-alloc}: For this case, given
    $\semenv : \Tenv \to \Val \to \Prop$ and a list of values
    $\vv{\val}$ such that
    $\semGtype{\Tenv}{\env}{\semenv}(\vv{\val})$, we need to show
    assuming
    \begin{align}
      \wpre{\subst{\expr}{\vv{\var}}{\vv{\val}}}{\semVtype{\Tenv}{\typ}{\semenv}} \label{eqn:unary-Tref-alloc-rel}
    \end{align}
    that the following holds:
    \[ \wpre{\Ref(\subst{\expr}{\vv{\var}}{\vv{\val}})}{\var.  \exists
        \loc.\; \var = \loc \land \knowInv{\namesp.\loc}{\exists
          \valB.\; \loc \mapsto \valB \ast
          \semVtype{\Tenv}{\typ}{\semenv}(\valB)}} \]

    We use the rule~\ruleref{wp-bind} together with the assumption
    \eqref{eqn:unary-Tref-alloc-rel} above. Consequently, we need to
    show that given some arbitrary value $\val$ such that
    \begin{align}
      \semVtype{\Tenv}{\typ}{\semenv}(\val) \label{eqn:unary-Tref-pointsto}
    \end{align}
    we have
    \[\wpre{\Ref(\val)}{\var.  \exists \loc.\; \var = \loc \land
        \knowInv{\namesp.\loc}{\exists \valB.\; \loc \mapsto \valB
          \ast \semVtype{\Tenv}{\typ}{\semenv}(\valB)}} \]
    We proceed by applying the rule~\ruleref{wp-alloc} which requires
    us to show:
    \[
      \later \forall \loc'.\; \loc' \mapsto \val \wand
      \wpre{\loc'}{\var.  \exists \loc.\; \var = \loc \land
        \knowInv{\namesp.\loc}{\exists \valB.\; \loc \mapsto \valB
          \ast \semVtype{\Tenv}{\typ}{\semenv}(\valB)}}
    \]
    This follows easily from the rules~\ruleref{wp-val} and
    \ruleref{Inv-alloc} together with assumption
    \eqref{eqn:unary-Tref-pointsto} above.

  \item[--] Case \ruleref{T-rec}: For this case, given
    $\semenv : \Tenv \to \Val \to \Prop$ and a list of values
    $\vv{\val}$ such that
    $\semGtype{\Tenv}{\env}{\semenv}(\vv{\val})$, we need to show
    assuming
    \begin{align}
      \forall \val, \valC.\; \semVtype{\Tenv}{\typ}{\semenv}(\val) \ast
      \semVtype{\Tenv}{\typ \to \typ'}{\semenv}(\valC) \wand
      \wpre{\subst{(\subst{\expr}{\vv{\var}}{\vv{\val}})}{\var, f}{\val,
      \valC}}{\semVtype{\Tenv}{\typ'}{\semenv}} \label{eqn:unary-body-rel}
    \end{align}
    that the following holds:
    \[\wpre{\Rec f x = \subst{\expr}{\vv{\var}}{\vv{\val}}}{\var.\;
        \persistently\left(\forall \valB.\;
          \semVtype{\Tenv}{\typ}{\semenv}(\valB) \wand
          \semEtype{\Tenv}{\typ'}{\semenv}(\var~\valB)\right)}\]
    By the rule~\ruleref{wp-val} it suffices to show:\footnote{We
      can introduce the persistence modality because all assumptions
      are persistent.}
  \[\forall w.\; \semVtype{\Tenv}{\typ}{\semenv}(\valB) \wand
    \semEtype{\Tenv}{\typ'}{\semenv}((\Rec f x =
    \subst{\expr}{\vv{\var}}{\vv{\val}})~\valB)\] Note the operational
  semantics pertaining to calling a recursive functions. The
  expression
  \[(\Rec f x = \subst{\expr}{\vv{\var}}{\vv{\val}})~\valB\] reduces
  to the expression
  \[\subst{(\subst{\expr}{\vv{\var}}{\vv{\val}})}{\var, f}{\valB, \Rec
    f x = \subst{\expr}{\vv{\var}}{\vv{\val}}}\] in a single step of
computation. Hence, if we know that
$\semVtype{\Tenv}{\typ \to \typ'}{\semenv}(\Rec f x =
\subst{\expr}{\vv{\var}}{\vv{\val}})$ holds we can use the assumption
\eqref{eqn:unary-body-rel} above to finish the proof. However, this is
exactly what we have to show but crucially we only need this
\emph{after one step of computation}, intuitively because a recursive
call can only occur inside the body, after the current call. Hence, to
finish the proof we use the~{L{\"o}b}
rule. Consequently we get to assume the following L\"ob induction
hypothesis \eqref{eqn:unary-loeb-ind-hyp-in-proof-of-Trec}.
  \begin{align*}
    \later \forall w.\; \semVtype{\Tenv}{\typ}{\semenv}(\valB) \wand
    \semEtype{\Tenv}{\typ'}{\semenv}((\Rec f x =
    \subst{\expr}{\vv{\var}}{\vv{\val}})~\valB)
    \tag{IH}\label{eqn:unary-loeb-ind-hyp-in-proof-of-Trec}
  \end{align*}
  We finish the proof using the rule~\ruleref{wp-rec} which requires
  us to show the following for some arbitrary value $\valB$ for which
  we have $\semVtype{\Tenv}{\typ}{\semenv}(\valB)$:
  \[ \later \wpre{\subst{(\subst{\expr}{\vv{\var}}{\vv{\val}})}{\var,
        f}{\valB, \Rec f x = \subst{\expr}{\vv{\var}}{\vv{\val}}}}
    {\semVtype{\Tenv}{\typ'}{\semenv}} \] This follows easily from our
  assumptions and the L\"ob induction hypothesis,
  \eqref{eqn:unary-loeb-ind-hyp-in-proof-of-Trec}, above.
\item[--] Case \ruleref{T-tlam}: For this case, given
  $\semenv : \Tenv \to \Val \to \Prop$ and a list of values
  $\vv{\val}$ such that $\semGtype{\Tenv}{\env}{\semenv}(\vv{\val})$,
  we need to show assuming
    \begin{align}
      \forall \predB.\; \semGtype{\Tenv, \tvar}{\env}
      {\semenv, \tvar \mapsto \predB}(\vv{\val}) \proves
      \wpre{\subst{\expr}{\vv{\var}}{\vv{\val}}}{\semVtype{\Tenv, \tvar}{\typ'}
      {\semenv, \tvar \mapsto \predB}}
      \label{eqn:unary-tlam-body-rel}
    \end{align}
    that the following holds:
   \begin{align*}
     \wpre{\subst{\TLam \expr}{\vv{\var}}{\vv{\val}}}{\var.\;
     \persistently\left(\forall \predB.\; \persistent{\predB} \Rightarrow
     \semEtype{\Tenv, \tvar}{\typ}{\semenv, \tvar \mapsto \predB}(\val~\_)\right)}
   \end{align*}

   Since $\TLam \subst{\expr}{\vv{\var}}{\vv{\val}}$ is a value, we
   use the rule~\ruleref{wp-val}. Hence, it suffices to show the
   following for some arbitrary but fixed $\predB$ such that
   $\persistent{\predB}$:
   \[\semEtype{\Tenv, \tvar}{\typ}{\semenv, \tvar \mapsto\predB}
     ((\TLam \subst{\expr}{\vv{\var}}{\vv{\val}})~\_)\] Note that here
   we can introduce the persistence modality as none of our
   assumptions assert any ownership. Unfolding the expression relation
   in the above formula reveals that we need to show:
   \[\wpre{(\TLam \subst{\expr}{\vv{\var}}{\vv{\val}})~\_}
     {\semVtype{\Tenv, \tvar}{\typ}{\semenv, \tvar \mapsto \predB}}
   \]
   To prove this, we proceed by applying the rule~\ruleref{wp-Tlam}
   and as a result need to show:
   \[\later \wpre{\subst{\expr}{\vv{\var}}{\vv{\val}}}
     {\semVtype{\Tenv, \tvar}{\typ}{\semenv, \tvar \mapsto \predB}}\]
   Finally, we can finish the proof by appealing to assumption
   \eqref{eqn:unary-tlam-body-rel}. We only need to show
   $\semGtype{\Tenv, \tvar}{\env}{\semenv, \tvar \mapsto
     \predB}(\vv{\val})$ while we have
   $\semGtype{\Tenv}{\env}{\semenv}(\vv{\val})$. However, this follows
   from Lemma~\ref{lem:unary-interp-env-weaken}.
\end{itemize}
\end{proof}


\begin{lemma}[Adequacy of unary logical relations]\label{thm:unaryadequacy}
  Let $\expr$ be an expression such that
  $\semEtype{\Tenv}{\typ}{\semenv}(\expr)$. Then, $\expr$ is safe,
  $\SAFE(\expr)$.
\end{lemma}

\begin{proof}
  This lemma is a direct consequence of the adequacy theorem, see~\cite{iris-ground-up}.
\end{proof}

\begin{theorem}[Soundness of unary logical relations] All closed
  well-typed programs of \TheLang\, are safe:
  \[\text{If }~ \typed{\cdot}{\cdot}{\expr}{\typ} ~\text{ then }~ \SAFE(\expr) \]
\end{theorem}

\begin{proof}
  By the fundamental theorem of unary logical relation,
  Theorem~\ref{thm:unaryfundamental}, we know that
  \[ \semtyped{\cdot}{\cdot}{\expr}{\typ} \]
  Expanding the definition of unary logical relations we get
  \[ \forall \semenv, \vv{\val}.\;
    \semGtype{\cdot}{\cdot}{\semenv}(\vv{\val}) \proves
    \semEtype{\cdot}{\typ}{\semenv}\left(\subst{\expr}{\vv{\var}}{\vv{\val}}\right) \]
  We take $\semenv = \emptyset$ and $\vv{\val} = \epsilon$ which gives us
  \[ \semGtype{\cdot}{\cdot}{\emptyset}(\epsilon) \proves
    \semEtype{\cdot}{\typ}{\semenv}\left(\subst{\expr}{\epsilon}{\epsilon}\right) \]
  which immediately simplifies to
  $\semEtype{\cdot}{\typ}{\semenv}(\expr)$. By
  Theorem~\ref{thm:unaryadequacy}, we get $\SAFE(\expr)$, as required.
\end{proof}

The logical relations model presented in this section is modular. For
instance, a well-typed function of type $\typ \to \typ'$ (which by the
fundamental theorem falls in the logical relation) can be applied to
any expression that is the logical relations for $\typ$ and the result
is in the logical relation for $\typ'$. On the other hand, our logical
relations model is defined in terms of untyped expressions. This means
that our logical relations model can be used to prove safety of
programs that mix well-typed code with untyped code as long as we show
that the untyped code is \emph{semantically well-typed}, \ie\ the
untyped code is in the logical relations for the appropriate type. As
an example of a program that is \emph{not} well-typed but is
nonetheless semantically well-typed consider the following:
\begin{align*}
  \TLam \TLam (\Lam f. \Lam \var. \Lam g. \Lam \varB.
  & \Let l = \Ref(\True) in \Fork{l \gets \False}; \\
  & \If \deref l then \operatorname{waitfor}~l; l \gets \var; \Inj{1} (f~l) \Else l \gets \varB; \Inj{2}(g~l))
\intertext{where}
& \operatorname{waitfor} \eqdef \Rec f \var = \If \deref \var then f~\var \Else \TT
\end{align*}

This program does not syntactically have the type
\[\Tall \tvar. \Tall \tvarB. (\Tref(\tvar) \to \tvar) \to \tvar
  \to (\Tref(\tvarB) \to \tvarB) \to \tvarB \to \tvar + \tvarB\] 
  but
it \emph{semantically} does. This program allocates a boolean
reference and uses it to non-determi-nistically call $f$ or $g$. In
each case it changes the reference that it has already allocated with
the given value of the appropriate type before passing it to the
chosen function. In case the decision is made to call the first
function, \ie\ the other thread has not succeeded in the race, it
waits for the other thread to finish. This is to ensure that the other
thread writing to the reference $l$ is not going to destroy the
contents of $l$ at some later point. Despite not being syntactically
well-typed one can show that the program above is semantically of the
type given. Hence, this program can be \emph{safely} linked against
any other (syntactically or semantically) well-typed program with
compatible type. 

\subsection{Binary logical relation for contextual refinement}
\label{sec:binary-logical-relation}

In~\cite{timany:logrel-in-iris} the authors also define a binary logical relations model for \TheLang\ and prove that logical relatedness implies contextual refinement.

Many of the ideas from the unary case carry over to the binary case, \eg\ the binary value interpretation is a straightforward modification of the unary value interpretation. 
The expression interpretation, however, is quiet different: in the unary case, it sufficed to use a simple weakest precondition definition, but now, in the binary case,
the challenge is to find a way to relate two (different) expressions.
In this section, we sketch the key ideas of how one can define a relational model. We do not include the full definition of the binary logical relation, but instead refer the
reader to~\cite{timany:logrel-in-iris}.

In the relational model, the goal is to define a logical relation, such that if $\expr$ is logically related to $\expr'$ then $\expr$ contextually refines $\expr'$.
We often refer to the expression ``on the left'', $\expr$, as the implementation side expression and the expression ``on the right'', $\expr'$, as the specification side expression. 
To define the relational models, we need a way to 
refer to (the heap and different threads of) the program on the
specification side ``that is about to be executed''. The intuitive
definition of two expressions being related is as follows:
\begin{quote}
  \emph{
  An expressions $\expr$ (the
  implementation side) is related to an expression $\expr'$ (the
  specification side) if we have:
  \begin{align*}
    \forall j, \lctx.\;
    & \text{``thread $j$ is about to execute $\lctx[\expr']$''}~ \wand\\
    & \wpre{\expr}{\exists \val'.\; \text{``thread $j$
      is about to execute $\lctx[\val']$''}}
  \end{align*}
}
\end{quote}
%
This relation between $\expr$ and $\expr'$ reads as follows: if thread
$j$ is about to execute $\expr'$ under some evaluation context $\lctx$
on the specification side and $\expr$ reduces to a value, then there
is a value $\val'$ such that the specification side is about to
execute $\lctx[\val']$. In other words, whenever $\expr$ reduces to a
value we know that $\expr'$ has also been reduced to $\val'$. The
reason for explicit quantification over the thread $j$ under which the
specification side is being executed is to enable thread-local
reasoning. We quantify over the evaluation context $\lctx$ under which
the expression is about to be executed to enable modular reasoning
with respect to evaluation contexts.


The goal is now to be able to formalize that "thread $j$ is about to execute $\lctx[\expr']$". For this, we first define the monoids:
\begin{align*}
\textsc{Heap} \eqdef{}&\authm(\Loc \fpfn (\exm(\Val))) \\
\textsc{Tpool} \eqdef{}& \authm(\nat \fpfn (\exm(\Expr))) \\
\end{align*}
Letting $\gname_h'$, $\gname_{\mathit{tp}}'$ be instances of \textsc{Heap}\footnote{Notice that is the same resource algebra, that is used to define the standard $\mapsto$ predicate. Hence we now have two instances of it, one $\gname_h$ for tracking the heap on the implementation side and one $\gname_h'$ for tracking the heap on the specification side.} respectively \textsc{Tpool} we now define the following propositions:
\begin{align*}
  \SpecConf(\state, \vv{\expr}) \eqdef{}
  & \ownGhost{\gname_h'}{\authfull \mathrm{res}(\state)} \ast
    \ownGhost{\gname_{\mathit{tp}}'}{\authfull \mathrm{fpfnOf}(\vv{\expr})}\\
  \loc \mapsto_s \val \eqdef{}& \ownGhost{\gname_h'}{\authfrag [\loc \mapsto \val]}\\
  j \Mapsto_s \expr \eqdef{}& \ownGhost{\gname_{\mathit{tp}}'}{\authfrag [j \mapsto \expr]}\\
\intertext{where}
\mathrm{fpfnOf}(\expr_1, \dots, \expr_n) \eqdef{}& \set{(i, \expr_i) \middle| 1 \le i \le n}
\end{align*}
Here, $\cfgg$ is a configuration, \ie\ a pair of a heap and a thread
pool, $(\state, \vv{\expr})$. The intuition is that $\SpecConf(\cfgg)$
is the (unique) configuration that the specification side is in. 
The propositions $j \Mapsto \expr$ and $\loc \mapsto_s \val$ simply
specify the exclusive ownership of the execution of a thread or a
memory location on the specification side. Naturally, this requires the propositions to be exclusive.

Furthermore we use the following invariant to express that the specification side is in a configuration reachable from some starting configuration:

\[\SpecCtx(\cfgg) \eqdef{} \knowInv{\namesp.sc}{\exists \cfgg'.\;
    \SpecConf(\cfgg') \land \cfgg \tpstep ^{*} \cfgg'} \]

With these tools, it is possible to define the expression relation as:

\begin{align*}
  \semEtype{\Tenv}{\typ}{\semenv}(\expr, \expr') \eqdef{}
  & \forall \cfgg, j, \lctx.\; \SpecCtx(\cfgg) \ast j \Mapsto \lctx[\expr'] \wand\\
  & \wpre{\expr}{\val.\;
    \exists \val'.\; j \Mapsto \lctx[\val'] \ast \semVtype{\Tenv}{\typ}{\semenv}(\val, \val')} \\
\end{align*}
The expression relation above states that $\expr$ and $\expr'$ are
related if the following holds: for any thread $j$ which is about to
execute $\expr'$ under some evaluation context $\lctx$, it is safe to
evaluate $\expr$ and whenever $\expr$ reduces to a value $\val$, we
know that $\expr'$ has also been evaluated to some value $\val'$ in
thread $j$ under the evaluation context $\lctx$. Furthermore, we know
that $\val$ and $\val'$ will be related as values of the type relating
$\expr$ and $\expr'$. Thus, essentially, two expressions $\expr$ and
$\expr'$ are related at type $\typ$ if, whenever $\expr$ reduces to a
value, so does $\expr'$ (no matter under which circumstances it is
being evaluated), and the resulting values will be related at type
$\typ$.

One can now define the binary logical relations for \TheLang{}, written
$\semtyped{\Tenv}{\env}{\expr \lelog \expr'}{\typ}$, as follows:
\[\semtyped{\Tenv}{\env}{\expr \lelog \expr'}{\typ} \eqdef{} \forall
  \vv{\val}, \vv{\val'}, \semenv.\;
  \semGtype{\Tenv}{\env}{\semenv}(\vv{\val}, \vv{\val'}) \proves
  \semEtype{\Tenv}{\typ}{\semenv}(\subst{\expr}{\vv{\var}}{\vv{\val}},
  \subst{\expr'}{\vv{\var}}{\vv{\val'}}) \] where $\vv{\var}$ is the
domain of $\env$.

See~\cite{timany:logrel-in-iris} for the definition of the value relation and also
for applications of the logical relation for proving challenging
contextual refinements.

%%% Local Variables:
%%% mode: latex
%%% TeX-master: "../main.tex"
%%% End:
