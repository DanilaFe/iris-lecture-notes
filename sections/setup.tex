\section{Programming Language}
\label{sec:setup}

A \emph{program} logic is used to reason about \emph{programs}, that is, to specify their behaviour.
The precise rules of the logic depend on the constructs present in the
programming language.
Iris is really a framework, which can be instantiated to a range of
different programming languages, but in order to learn about Iris,
it is useful first to get some experience with one particular choice of
programming language.
Thus in this section we fix a concrete programming language, which we will use throughout the notes.
Our language of choice, denoted  \proglang,
is an untyped higher-order ML-like language with general references and
concurrency.
Concurrency is supported via a primitive, $\Fork{e}$, for forking a
new thread to compute $e$, and
via a primitive compare and set, $\CAS$, operation.
This is the only synchronization primitive in the language.
Other synchronization constructs, such as locks, semaphores, etc., can be defined in the language, and indeed we implement and give specifications to two different kinds of locks.
The syntax and the operational semantics is shown in
Figure~\ref{fig:opsem}.

\paragraph*{Syntax sugar}
In addition to the given constructs we will use additional syntax sugar when writing examples.
We will write $\lambda x . e$ for the term $\Rec{f} x = e$ where $f$ is some fresh variable not appearing in $e$.
Thus $\lambda x . e$ is a non-recursive function with argument $x$ and body $e$.
We will also write $\Let x = e_1 in e_2$ for the term $(\lambda x . e_2) e_1$.
This is the standard let expression, whose operational meaning (see the operational semantics below) is to evaluate the term $e_1$ to a value $v$, and then evaluate $e_2[v/x]$, \ie{} the term $e_2$ with the value $v$ substituted for the variable $x$.
Referring to function definitions can be quite verbose, \eg\
$\langkw{loop} \mdef \Rec \textlang{loop} () = ();\ \textlang{loop}\ ()$.
As such, we often collapse the outer-most binding with the function binding,
and simply write $\langkw{loop}\ () = ();\ \langkw{loop}\ ()$.

\newcommand{\opsem}{
\textbf{Syntax}
\begin{displaymath}
  \begin{array}{lrcl}
&  x, y, f & \in{} & \textdom{Var} \\
%
& \loc & \in{} & \Loc \\
%
& n & \in{} & \integer \\
%
& \binop & \bnfdef{}& \Plus \ALT \Minus \ALT \Mult \ALT \Eq \ALT \Lt \ALT   \cdots\\
%
\Val\quad
& \val & \bnfdef{}&
 \TT
 \ALT \True
 \ALT \False
 \ALT n
 \ALT \loc
 \ALT (\val, \val)
 \ALT \Inj{1} \val
 \ALT \Inj{2} \val
 \ALT \langkw{rec}\spac f\ (\var) \mathrel{\ldef} \expr
 %% \ALT \Rec f \var = \expr         This gives an errors for some reason
 \\
\Expr \quad
& \expr & \bnfdef{} &
 \var
 \ALT n
 \ALT \expr \binop \expr 
 \ALT \TT
 \ALT \True
 \ALT \False
 \ALT \If \expr then \expr \Else \expr
 \ALT \loc \\ &&
 \ALT & (\expr, \expr)
 \ALT \Proj{1} \expr
 \ALT \Proj{2} \expr
 \ALT \Inj{1} \expr
 \ALT \Inj{2} \expr
 \ALT \Match{\expr}with{\Inj{1}x}=>{\expr}|{\Inj{2}y}=>{\expr}end \\&&
 \ALT & \langkw{rec}\spac f\ (\var) \mathrel{\ldef} \expr
 % \ALT \Rec f \var = \expr         This gives an errors for some reason
 \ALT \expr~\expr \\&&
 \ALT & \Ref(\expr)
 \ALT \deref \expr
 \ALT \expr \gets \expr
 \ALT \CAS(\expr,\expr,\expr)
 \ALT \Fork{\expr}\\
%
\ECtx\quad
& E & \bnfdef{}& 
 - 
 \ALT E \binop e 
 \ALT v \binop E 
 \ALT \If{E}then{e}\Else{e}
 \ALT (E,e) 
 \ALT (v,E)
 \ALT \Proj{1}{E}
 \ALT \Proj{2}{E}
 \ALT \Inj{1}{E} 
 \ALT \Inj{2}{E} 
 \\ &&
 \ALT & \Match{E}with{\Inj{1}x}=>{\expr}|{\Inj{2}y}=>{\expr}end 
 \ALT E\,\expr 
 \ALT v\,E
 \ALT \Ref(E) 
 \ALT \deref E 
 \ALT E \gets \expr 
 \ALT v \gets E\\  &&
 \ALT & \CAS(E,\expr,\expr') 
 \ALT \CAS(v,E,\expr) 
 \ALT \CAS(v,v',E) \\
%
  \Heap \quad  & h  & \in{} & \Loc\fpfn\Val \\
  \TPool \quad  & \El & \in{} & \nat\fpfn\Expr \\
  \Config \quad & \varsigma & \bnfdef{} & (h,\El) \\
\end{array}
\end{displaymath}
\textbf{Pure reduction}
\begin{align*}
  v \binop v' & \stepstopure v'' & & \text{if $v''= v \binop v'$} \\
  \If{\True}then{e_1}\Else{e_2} &\stepstopure e_1\\
  \If{\False}then{e_1}\Else{e_2} &\stepstopure e_2\\
  \Proj{i}(v_1,v_2) &\stepstopure v_i\\
  \Match{\Inj{i}v}with{\Inj{1}x_1}=>{e_1}|{\Inj{2}x_2}=>{e_2}end &\stepstopure e_i[v/x_i]\\
  (\Rec{f} x = e)\, v &\stepstopure e[(\Rec{f} x = e) / f, v / x]
\end{align*}
\textbf{Per-thread one-step reduction}
\begin{align*}
  (h,e) & \stepsto (h,e')
  &&\text{if $e\stepstopure e'$} \\
  (h, \Ref(v)) &\stepsto (h[\ell \mapsto v], \ell) 
  & &\text{if $\ell \not \in \dom(h)$}\\
  (h, \deref \ell) &\stepsto (h, h(\ell)) 
    & &\text{if $\ell \in \dom(h)$}
\\
  (h, \ell \gets v) &\stepsto \left(h[\ell \mapsto v], \TT\right) 
    & &\text{if $\ell\in\dom(h)$}
\\
  (h,\CAS(\ell,v_1,v_2)) & \stepsto \left(h[\ell\mapsto  v_2],\True\right)
 && \text{if $h(\ell) = v_1$} \\
(h,\CAS(\ell,v_1,v_2)) & \stepsto \left(h,\False\right)
 &&  \text{if $h(\ell) \neq v_1$}
\end{align*}
\textbf{Configuration reduction}
\begin{mathpar}
  \infer{(h, e) \stepsto (h', e')}
  {(h, \El[i\mapsto E[e]]) \cstepsto (h', \El[i \mapsto E[e']])}
  \and
  \infer{j\notin \dom(\El)\cup\{i\} }
  {(h, \El[i\mapsto E[\Fork{e}]]) \cstepsto (h, \El[i \mapsto
    E[()][j\mapsto e]])}
\end{mathpar}
}

%
\begin{figure}[htbp]
  \opsem
  \caption{Syntax and Operational Semantics of \proglang.}
  \label{fig:opsem}
\end{figure}

\paragraph*{Operational semantics}
The operational semantics is defined by means of pure reductions,
one-step reductions involving the heap, and general reductions 
among configurations.  
A configuration consists of a heap and a thread
pool, and a thread pool is a mapping from thread identifiers (natural
numbers) to expressions, i.e., a finite set of named threads.
Note that reduction of configurations is
nondeterministic: we may choose to reduce in any thread
in the thread pool.
This reflects that we are modelling a kind of preemptive concurrent system.
Any thread can be suspended at any time while some other thread gets to run.
Further note that reduction of a $\Fork{e}$ expression in thread $i$ 
proceeds by creating a new thread, with thread identifier $j$, whose initial expression is $e$,
and that the result of evaluation $\Fork{e}$ is the unit value $()$.
Evaluation contexts are used to specify the evaluation strategy, that is,
where the next reduction in a
thread may take place.  We use a call-by-value, left-to-right,
evaluation strategy.
Left-to-right refers to the evaluation of function applications, pairs, and binary operations, assignment, and compare and set $\CAS$.
For example, the pair $(e_1,e_2)$ is evaluated by first evaluating $e_1$ (the \emph{left}most term), and then $e_2$.
We finally remark on the compare and set  $\CAS$ primitive: in 
a heap $h$, compare and set $\CAS(\ell,v,v')$ \emph{atomically},
that is, \emph{in one reduction step}, looks up the value of $\ell$ in
$h$, compares it to $v$, and if it equals $v$, updates $h(\ell)$ to
contain $v'$.
(On real machines, $\CAS$ may only be used to compare values that fit
in a machine word, such as pointers and integers -- for simplicity, we
do not formalize such a restriction here; see~\cite{logrel-fcd} for
an example of how it may be done.)

Let us illustrate an execution of a simple concurrent program.
The program allocates a new location with initial value $0$ and creates a new thread which updates the location to $3$ and terminates.
The main, \ie{} original, thread waits until the location has been changed from $0$ at which point it reads it and adds $1$ to it.
The program is written as follows.
\begin{align*}
  &\Let x = \Ref(0) in\\
  &\Let y = \Fork{\CAS(x,0,3)} in\\
  &(\Rec{f} {} = \If \deref x = 0 then f \TT \Else x \gets \deref{x} + 1) \TT
\end{align*}
Let us call it $e$.
\begin{exercise}
  Come up with at least two different executions of this program.
\end{exercise}


%%% Local Variables:
%%% mode: latex
%%% TeX-master: "../main.tex"
%%% End:
